
\documentclass[12pt]{article}
\usepackage{geometry}
\usepackage[hyphens]{url} % see geometry.pdf on how to lay out the page. There's lots.
\geometry{a4paper} % or letter or a5paper or ... etc
% \geometry{landscape} % rotated page geometry

% See the ``Article customise'' template for come common customisations

\title{Comp 116 Final Paper Outline}
\author{Stefan Dimitrov}
%\date{} % delete this line to display the current date

%%% BEGIN DOCUMENT
\begin{document}

\maketitle

\begin{description}
  \item[Introduction] \hfill \\ Problem: iOS jailbreaking has been stigmatized as a potential invitation for malware and security risks, while too much trust is placed in the walled garden model of software distribution through the App Store. How can we turn jailbreaking into an advantage?

  \item[To the Community] \hfill 
  \begin{enumerate}
 \item  Jailbreaking allows for:
  		\begin{itemize}
		\item unrestricted security research on the iOS platform.
		\item third party patches in response to security vulnerabilities.
		\end{itemize}
 	 \item 	This paper:
		 \begin{itemize}
		\item points out some security and privacy benefits of jailbreaking
		\item uncovers pitfalls that could make jailbreaking a security risk
		\end{itemize}
	\end{enumerate}
	
 \item[Privacy issues] \hfill 
   	\begin{enumerate}
	\item UDID leaks [b]
	\item Path address book fiasco
	\item Apps sharing too much for no reason other than identification, data collection, etc.
	\item \ldots
	\end{enumerate}
 \item[Security issues of stock-iOS devices] \hfill 
    	\begin{enumerate}
	\item Methods for subverting the signing process on non jailbroken devices -  a GBA emulator?s source code used to be available on github and could be compiled and installed without a developer account. ad hoc distribution using leaked udids
	\item Methods for injecting obfuscated malicious code into an App Store app
	\item Other methods of delivering malware
	\item \ldots
	\end{enumerate}

 \item[Security issues of jailbreaken devices] \hfill 
     	\begin{enumerate}
	\item Malicious/untrusted repositories
	\item SSH attacks with default root password
	\item Retaining a jailbreak means not updating to the latest iOS
	\item \ldots
	\end{enumerate}

 \item[Security benefits of jailbreaking] \hfill 
     	\begin{enumerate}
	\item Delivery of patches (jailbreak.me fix for pdf vulnerability)
	\item Allows for security research, finding flaws and patching them
	\item Enhancing privacy - on stock iOS7 camera access still unrestricted, ad location tracking [a]
	\item Prevents \emph{other} users from jailbreaking for malicious purposes (bypassing the lockscreen passcode lock for example) [c]
	\item \ldots
	\end{enumerate}
	
 \item[Summary] \hfill 
 

\item [References] 

\item [iSAM: An iPhone Stealth Airborne Malware] \hfill \\ \url{http://www.icsd.aegean.gr/publication_files/conference/62773319.pdf}
\item [PiOS: Detecting Privacy Leaks in iOS Applications] \hfill \\ \url{http://seclab.cs.ucsb.edu/media/uploads/papers/egele-ndss11.pdf}
\item [Jekyll on iOS: When Benign Apps Become Evil] \hfill \\ \url{http://www.cc.gatech.edu/~klu38/publications/security13.pdf}
\item [A survey of mobile malware in the wild] \hfill \\ \url{http://dl.acm.org/citation.cfm?id=2046618}
\item [Exploiting the iOS Kernel] \hfill \\ \url{http://media.blackhat.com/bh-us-11/Esser/BH_US_11_Esser_Exploiting_The_iOS_Kernel_WP.pdf}

\item [Articles] \hfill \\
  \texttt{[a]} \url{http://ios.wonderhowto.com/how-to/18-sneaky-privacy-betraying-settings-every-iphone-owner-must-know-about-ios-7-0148682/}  \\\\
 \texttt{[b]} \url{http://www.crowdstrike.com/blog/finspy-mobile-ios-and-apple-udid-leak/index.html} \\\\
 \texttt{[c]} \url{http://www.zdnet.com/ios-7-apples-war-against-jailbreaking-now-makes-perfect-sense-7000016623/} 

\end{description}


\end{document}