\documentclass[12pt, titlepage, oneside]{article}
%\usepackage{amsaddr}
\usepackage{setspace}
\usepackage{geometry}                % See geometry.pdf to learn the layout options. There are lots.
\geometry{letterpaper}                   % ... or a4paper or a5paper or ... 
%\geometry{landscape}                % Activate for for rotated page geometry
%\usepackage[parfill]{parskip}    % Activate to begin paragraphs with an empty line rather than an indent
\usepackage{graphicx}
\usepackage{amssymb}
\usepackage{hyperref}
\usepackage{epstopdf}


\newcommand{\tab}{\hspace*{2em}}
\DeclareGraphicsRule{.tif}{png}{.png}{`convert #1 `dirname #1`/`basename #1 .tif`.png}

\title{Privacy and Security Benefits of Jailbreaking iOS}
%\author{Stefan Dimitrov}

\author{
  Stefan Dimitrov\\
  \texttt{sdimit01@tufts.edu}
  %\and
  \\\\
  Ming Chow\\
  \emph{Mentor}
}

%\address{sdimit01@tufts.edu}
%\author{Mentor}
%\author{Ming Chow}
%\address{Mentor}

%\author{\today}

%\author{}
%\date{\today}                                           % Activate to display a given date or no date

%\pagebreak
\begin{document}

\maketitle

\begin{abstract}

The traditional software distribution channel on the iOS platform is Apple's own AppStore, which is known for its stringent app approval policies and quality control. One of the aims of Apple's approval process is to prevent malicious software from reaching its customer base. This preemptive strategy has proven to be relatively successful, but has been criticized for rejecting benign apps that replace or enhance core services. Jailbreaking has opened an alternative outlet for this niche of software offerings by letting users install apps from third party software repositories outside of Apple's regulation, however, this lack of security audit theoretically enables the unhindered distribution of potentially malicious code. Moreover, because jailbreaks exploit a security vulnerabilities within iOS in order to grant apps root privileges and unrestricted filesystem access, they effectively disable several security features in iOS. However, jailbreaking also allows the installation of third-party patches targeting those very vulnerabilities. In addition, jailbreak tweaks that improve privacy control have surfaced in response to controversial AppStore apps that misuse user information, yet pass the approval process. These subtle, but significant differences between the stock and jailbroken flavors of iOS motivate this paper in an attempt to explore the pros and cons of jailbreaking regarding security and privacy.

\end{abstract}.



\tableofcontents

\pagebreak
\doublespacing

\section{Introduction}
\tab The App Store is an integral part of the iOS security model. All applications on the App Store must go through an approval process, in part to prevent malicious apps from reaching users. This process involves quality control, static and manual analysis of the behavior and content of apps. In addition, the use of private APIs, which typically would allow an app to provide unapproved or malicious functionality is screened against. Each app on the iOS platform is assigned a unique identifier and a folder whose contents it can modify, while the rest of the filesystem is out of reach. Individual apps can communicate with other apps in a very limited fashion by means of URL Handlers. Access to Apple services such as Push Notifications, Game Center and Passbook is authenticated through certificates issued by Apple. Moreover, every application on the App Store is code signed, and code that hasn't been signed by Apple cannot run on the device. All these measures ensure that apps run in a sandboxed environment which ensures that the system and the user data on the device are protected in case an app attempts malicious activity.

Jailbreaking is the process of modifying iOS through what is most often a security exploit in order to allow code that hasn't been signed by Apple to run on the device, in addition to enabling root access to the filesystem for any app, effectively subverting the sandboxing on iOS. Since jailbreaking allows the execution of code that has not been signed by Apple, it makes it possible to download and install apps and tweaks from sources exempt of Apple's stringent approval procedures. Such apps and tweaks typically enhance the functionality of the OS and its stock apps or boast features that are impossible to implement with the official SDK, as jailbreaking permits developers to use officially unapproved private APIs that provide additional functionality.

\section{To the Community}
This paper is aimed at iOS researchers and advanced users, and seeks to dispel the notion that jailbreaking is fundamentally insecure. With this paper I hope to raise awareness about the dangerous sense of false security and privacy that comes with Apple's walled garden model of software distribution. 

Jailbreaking can be invaluable when it comes to security and privacy research on the iOS platform. Unrestricted access to the platform's internals and filesystem enables insight into security and privacy strengths and weaknesses, given that iOS is closed source. Moreover, jailbreaking allows researchers to run code unsigned by Apple, use Apple's private frameworks and even patch the OS itself, which can aid in discovering system vulnerabilities and developing fixes for them. 

Privacy and security enhancing software offerings in their essence must be able to modify or access low level system functionality in order to detect apps leaking information or misusing privileges and prevent this from happening. Since apps that access the system internals and private frameworks are strictly forbidden, such apps have only been available to the users of jailbroken iOS devices. 

Considering these important uses of jailbreaking, in the following sections I outline the findings of researchers regarding privacy and security on iOS, alongside recent cases of privacy flaws in iOS and apps from third-party developers. In the section Action Items I showcase useful tools that can improve the security and privacy of jailbroken iOS users and how they pertain to real threats.


\section{Sneaking malicious code into the AppStore}
\tab Prior to submitting an app each developer needs to enroll into the paid Apple Developer Program and disclose their identity. This allows Apple to ban developers caught committing infractions from submitting apps to the App Store. While this might discourage developers of malicious apps under the threat of receiving a permanent ban from the App Store, it would be unwise to believe in the benevolence of developers solely due to the existence of a disincentive to misbehaving. 

In reality, since the approval process does not require that developers submit their source code for a review, it is quite possible to use code obfuscation techniques in order to evade any static or dynamic code analysis tools Apple might be using to uncover threats.  In his paper on iOS privacy\cite{priv}, Nicolas Seriot demonstrates several simple techniques that can be used to deliberately circumvent common types of code analysis likely used by Apple to search for calls to private APIs and system directories. Seriot suggests that his proof-of-concept SpyPhone app\footnote{https://github.com/nst/spyphone/} in fact does not rely on private APIs in the first place, and goes on to express concern that the existence of apps on the App Store collecting personal data is quite likely given the underground demand for and abundance of personal data on iOS devices.

Another group of researchers from Georgia Institute of Technology present\cite{jekyll} their "Jekyll App" which obtained Apple's approval, despite allowing the researchers to remotely launch a variety of attacks on their own tests devices. The paper demonstrates in practice how it is possible to create malicious apps that evade detection by breaking down malicious code into distinct subroutines termed "gadgets" which are not malicious on their own but if  \emph{remotely} supplied the correct execution paths can be assembled into a malicious routine. Breaking down code with malicious effects into innocuous looking subroutines practically renders static analysis less useful as the malicious code no longer appears explicitly. Moreover, since the malicious routines are enabled remotely, even the effectiveness of manual testing and dynamic analysis is diminished. 

%\cite{attacks}, 
\section{Threats to Privacy in the AppStore}
Even though researchers have demonstrated techniques such as dynamic \cite{dynamic} and static \cite{pios} analysis targeted at capturing data flows within the application have been developed, several controversial cases of user data misuse and user profiling have gained attention\footnote{http://mclov.in/2012/02/08/path-uploads-your-entire-address-book-to-their-servers.html}. 

A recent case of privacy violations emerged in November 2013, as the developer Kyle Richter discovered that a popular game app QuizUp transmitted sensitive information about a players Facebook friends, Address Book contacts and others, in a plain text format over to the company servers, presumably in an effort to match the user with other players.\footnote{http://kylerichter.com/our-responsibility-as-developers/} Despite the fact that thanks to the privacy enhancements in iOS, the app had to request users' permission to collect the data, the users are still not able to fine tune the permissions granted, for example by disallowing access to contacts' email addresses or real names, which wouldn't be of use to a game matching algorithm.

While AppStore apps can pose threats to privacy, iOS itself collects large amounts of data from normal usage.
In 2011, a controversy arose due to iOS maintaining a database of WiFi hotspots and cell towers around the users location in order to improve its geo-location services beyond complete reliance on the slower GPS systems which would allow for quickly determining the user's current location when using Maps.\footnote{http://www.networkworld.com/community/blog/apple-officially-responds-ios-consolidateddb-} Even though Apple assured that the location data is being sent to Apple servers in an anonymized manner, the data would remain on the device itself in a file named consolidated.db which could be easily retrieved.\footnote{https://www.sans.org/reading-room/whitepapers/forensics/forensic-analysis-ios-devices-34092}. 

A new feature in iOS7, called Frequent Locations relies on a similar location data storage, which contains the users most frequently visited locations. The idea behind the feature is to enable Next Destination, a user-end feature that lets Apple provide relevant commute information such as travel time to work.\footnote{http://www.idownloadblog.com/2013/10/20/how-to-use-frequent-locations-and-next-destination-features-in-ios-7/} Despite users being able to disable those features, they are enabled by default and the data collection is not apparent.

\section{Action Items}
\subsection{Jailbreaking for improved privacy control}

Before address book permissions were first added to iOS 6, the well publicized case of the social networking app Path drew attention to the issue of privacy in the AppStore. It was discovered that Path collected personal information from the user's address book and sent it to company servers without the users consent and it did not list this activity in the privacy agreement. The creators of Path have since removed the behavior and settled the ensuing FTC charges.\footnote{http://www.ftc.gov/news-events/press-releases/2013/02/path-social-networking-app-settles-ftc-charges-it-deceived} In response to this, a tweak for jailbroken iOS devices named ContactPrivacy appeared. It allowed the user to control access to the address book.

As of iOS7 Apple doesn't request the users permission whenever malicious apps request to use the came for capturing video or photos. There are proof of concept spyware apps that allow surreptitious camera captures\cite{espionage} By hooking the methods for requesting the camera, I developed a tweak for MobileSubstrate (the code injection platform on most jailbroken devices). The simple tweak, named CameraSentry, allows the user to deny camera access to malicious spy apps. 

\subsection{Changing the default root password}

Users with jailbroken devices are advised to change the default root password - the well known "alpine". Since an attacker can install a remote login utility such as SSH and gain full control over the device, in case a device is jailbroken by an attacker, the default password (if left unchanged) can be used to access the filesystem, bypassing any lock screen passwords the user might be using (passwords are stored in \texttt{{\textbackslash}var{\textbackslash}Keychain})\cite{lostpass}

\subsection{Installing an outbound connections firewall}
Since iOS doesn't come with a user configurable firewall, it is impossible for the end user to monitor the outbound connections that apps attempt. In the case of privacy leaking applications this is of utmost importance as the user can prevent the leaking of information to third party servers shall they notice any suspicious data transfers (e.g. a game connecting to a server different than Apple's Game Center servers) A solution (Firewall iP 2.0) for outbound connection monitoring and control can be downloaded from the Cydia jailbreak app.
\subsection{Disabling anonymous usage statistics}

Developers use code from third-parties to collect usage statistics for their apps just like websites use third party solutions like Google Analytics to obtain insight about their customer base. Installing a jailbreak tweak like PrivaCy\footnote{http://cydia.saurik.com/package/com.saurik.privacy} allows end users to effectively disable sending such statistics.

\subsection{Jailbreaking for security and privacy research}
In order to perform any sort of dynamic analysis on a device, root access or at least filesystem access is necessary. Apple does not grant those on unmodified iOS devices and even developers are not granted direct access to the filesystem of the device. Jailbreaking is useful here as it provides researchers with unrestricted access to the OS. 
A group of researchers from Vienna University of Technology and UC Santa Barbara\cite{dynamic}, developed a method for dynamic analysis that relies on a VNC server for the automation of UI manipulation which is necessary to make sure that all code is being run while testing. Since the App Store restrictions pose technical limitations to implementing VNC servers, only the dynamic analysis method developed can only be used on jailbroken devices. Nevertheless, even more traditional tools such as \texttt{gdb} cannot be installed on iOS unless a jailbreak is in place.

\section{Conclusion}
While the AppStore is well known for its stringent approval policies, in light of the security research and controversial cases presented in this paper it is clear that stringent approval policies alone are not sufficient to protect users. Better static and dynamic analysis techniques are being researched\cite{dynamic} thanks to the runtime access provided by jailbreaking and can be used to enhance the Apple's screening process. Moreover, since mobile devices are highly personal, they contain valuable personal data which makes any potential security risk a serious privacy risk first and foremost. In this sense security and privacy are tightly related on iOS and the problem of privacy is somewhat intractable even when knowingly trusting data to a third party because there are no guarantees about how the data will be used. However, it is important that platforms evolve towards more power to the users when \emph{deciding} whether to give their data in the first place, complemented by the ability to fine tune the flow of sensitive data. Overall the enhancements that jailbreaking brings provide more control over privacy and can help uncover security threats by allowing unhindered security research on Apple's otherwise source platform.
%\subsection{}
\pagebreak

\nocite{*} 

\bibliographystyle{plain}

\bibliography{bibliography}

\end{document}  